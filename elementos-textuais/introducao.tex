% \titleformat{\section}[wrap]
% {\normalfont\bfseries}
% {\thesection.}{0.5em}{}

% \titlespacing{\section}{12pc}{1.0ex plus .1ex minus .2ex}{1pc}

\chapter{Introdução}
\section[Tema]{Tema: {\normalfont{Inteligência Artificial}}}

\section[Delimitação do Tema]{Delimitação do Tema: {\normalfont{Uso de inteligência artificial no jogo de xadrez para computador.}}}

\section{Problema}

O problema que norteará esta pesquisa está ligado à investigação da atuação dos melhores algoritmos de Inteligência artificial
nos motores de jogos de xadrez para computador, visando procurar as melhores jogadas dentro do jogo a partir da representação
matemática do tabuleiro e suas peças.

\section{Objetivos}

\subsection{Objetivo Geral}
Investigar e comparar os melhores algoritmos de inteligência artificial nos motores de jogos de xadrez para computador,
classificando-os com diversos parâmetros, como velocidade, qualidade de movimento e porcentagem de vitorias,
com o mais básico sendo a porcentagem de vitórias, assim revelando no que cada motor pode dedicar-se para sua melhoria.

\subsection{Objetivos Específicos}
- Analisar a representação do tabuleiro e o valor das jogadas possíveis dentro do jogo

- Analisar as diferenças entre as implementações dos algoritmos de busca \textit{min-max}, \textit{alpha-beta pruning} e
redes neurais nos motores de jogos de xadrez para computador.

- A partir dos dados coletados comparar os algoritmos e os motores de jogos de xadrez para computador.

- Especificar  em que áreas ou algoritmos cada motor de xadrez pode melhorar.

% Tem como finalidade explicar para o leitor do que trata a pesquisa, apresentando, de maneira sucinta, o tema do trabalho e sua delimitação, a problematização, os objetivos, a justificativa, as hipóteses e variáveis \cite{andrade,koche,medeiros}.

% Pode-se, também, indicar os principais teóricos que fundamentaram a pesquisa e descrever brevemente os assuntos abordados nas demais seções do trabalho \cite{medeiros}.

% O texto deve ser justificado, digitado em fonte \textit{Times New Roman} ou Arial, tamanho 12 e espaçamento de 1,5 entre as linhas, com exceção das citações com mais de três linhas, notas de rodapé e paginação, que devem ser em fonte tamanho 10 e espaçamento simples (1,0).