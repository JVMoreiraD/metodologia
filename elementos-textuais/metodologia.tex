% \chapter{Título da Seção Primária}

% \begin{figure}[!h]
%     \centering
%     \Caption{\label{fig:exemplo-2}Exemplo de Figura}
%     \IFCEfig{}{
%         \fbox{\includegraphics[width=8cm]{figuras/logo-ifce}}
%     }{
%         \Fonte{Elaborado pelo autor}
%     }
% \end{figure}

% \begin{table}[!h]
%     \centering
%     \Caption{\label{tab:exemplo-2} Exemplo de Tabela}
%     \IFCEtab{}{
%         \begin{tabular}{cll}
%             \toprule
%             Texto             & Texto             & Texto             \\
%             \midrule \midrule
%             Texto texto texto & Texto texto texto & Texto texto texto \\
%             Texto texto texto & Texto texto texto & Texto texto texto \\
%             Texto texto texto & Texto texto texto & Texto texto texto \\
%             Texto texto texto & Texto texto texto & Texto texto texto \\
%             Texto texto texto & Texto texto texto & Texto texto texto \\
%             Texto texto texto & Texto texto texto & Texto texto texto \\
%             \bottomrule
%         \end{tabular}
%     }{
%         \Fonte{Elaborado pelo autor}
%     }
% \end{table}

% \begin{quadro}[!h]
%     \centering
%     \Caption{\label{qua:exemplo-3} Exemplo de Quadro}
%     \IFCEqua{}{
%         \begin{tabular}{|c|c|l|l|}
%             \hline
%             Texto             & Texto             & Texto             & Texto             \\
%             \hline
%             Texto texto texto & Texto texto texto & Texto texto texto & Texto texto texto \\
%             \hline
%             Texto texto texto & Texto texto texto & Texto texto texto & Texto texto texto \\
%             \hline
%             Texto texto texto & Texto texto texto & Texto texto texto & Texto texto texto \\
%             \hline
%         \end{tabular}
%     }{
%         \Fonte{Elaborado pelo autor}
%     }
% \end{quadro}

% \newpage

% \begin{center}
%     Exemplo de referência \\
%     Livro: \cite{knuth} \\ Dissertação: \cite{Maia2011} \\ Artigo: \cite{lamport1986latex}
% \end{center}

% \begin{center}
%     Exemplo de Alíneas com Número
% \end{center}
% \begin{alineascomnumero}
%     \item Texto texto texto.
%     \item Texto texto texto.
%     \item Texto texto texto.
%     \item Texto texto texto.
%     \item Texto texto texto.
% \end{alineascomnumero}

% \begin{center}
%     Exemplo de Alíneas com Ponto
% \end{center}
% \begin{alineascomponto}
%     \item Texto texto texto.
%     \item Texto texto texto.
%     \item Texto texto texto.
%     \begin{subalineascomponto}
%         \item Texto texto texto.
%         \item Texto texto texto.
%     \end{subalineascomponto}
% \end{alineascomponto}
\chapter{Metodologia}
Com base em Marconi e Lakatos (2001), como estaremos utilizando métodos matemáticos, lógicos e estatísticos para esta pesquisa,
determinamos que esta terá caráter quantitativo.

Para esta pesquisa, estaremos utilizando os motores de xadrez chamados \textit{Stockfish}, \textit{Leela Chess Zero} (LCZ),
\textit{RubiChess}, \textit{Nemorino}, \textit{Igel}, \textit{Xifos}, \textit{Laser}, \textit{Defenchess}, \textit{Andscacs},
\textit{Halogênio}, \textit{Arasan} e o \textit{Combusken}, todos de código aberto.

A comparação que faremos entre os motores vai ser pela sua velocidade em calcular as melhores jogadas, a qualidade de movimentos,
sua porcentagem de vitórias e por seu ranking nos campeonatos de motores online, sendo o mais famoso deles o \textit{Top Chess Engine Championship}.

Primeiro começaremos avaliando a implementação da representação do tabuleiro de cada um dos motores de xadrez selecionados,
descrevendo como é a estrutura de dados que guarda: as posições de todas as peças, a informação de qual dos jogadores
é a vez, o direito de \textit{roque}, a casa com possibilidades de captura \textit{en passant} e o número de movimentos relacionado
a regra de 50 movimentos. Também mediremos quão boa a representação do tabuleiro com base nas três principais
formas de representação: centrada nas peças, centrada nas casas e a híbrida.


Depois de analisarmos a base do motor, iremos calcular a velocidade mínima, média e de pior caso de cada um dos algoritmos
apontados neste trabalho, levando em conta as variáveis de quantidade de peças em jogo, possíveis movimentos de \textit{roque}
e captura \textit{en passant}. Além de a partir das referências já citadas e seus diferentes métodos de cálculo de avaliação
de movimentos, verificar quais dos algoritmos produzem os melhores movimentos.

Criaremos uma estrutura de teste para verificar a porcentagem de vitórias de cada motor de xadrez, investigando
a participação de cada algoritmo nesta porcentagem. Levaremos em conta o elo ou \textit{ranking} dos motores em campeonatos,
seus feitos já realizados, e possíveis erros e \textit{bugs} encontrados.

Por fim, analisaremos quais possíveis mudanças e efeitos diferentes algoritmos de comunicação entre dois motores
e uma interface podem produzir dentro dos fatores aos quais iremos comparar os motores.
