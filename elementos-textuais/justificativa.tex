\chapter{justificativa}

A Inteligência Artificial (IA), foi escolhida como tema deste trabalho por sua grande importância atual e futura no
desenvolvimento da humanidade, tendo aplicações claras em diversas áreas, como carros autônomos, assistentes digitais,
algoritmos de reconhecimento de imagens, assim como qualquer projeto ou área que envolva a análise de uma grande base de
dados.

Optamos por analisar um tópico simplificado dentro do tema de Inteligência Artificial, que é o uso da mesma em algoritmos
utilizados em motores do jogo de xadrez para computador, com o intuito de investigar e comparar os algoritmos e motores para
classificá-los, revelando quais os melhores entre eles em diferentes quesitos e o porquê de assim serem, deste modo
apresentaremos em quais áreas cada motor e implementação de algoritmos podem empenhar-se para seu aperfeiçoamento.

O xadrez é um jogo entre dois oponentes, um de cada lado de um tabuleiro de 64 casas de cores alternadas, cada jogador
tem 16 peças. O objetivo é fazer um xeque mate, que acontece quando o rei está posicionado de forma que é possível
fazer sua captura e não é possível escapar.

A importância dos jogos de tabuleiro no tema é exposta por Luger (2013),
\begin{citacao}
    (...) os jogos de tabuleiro tem certas propriedades que os tornaram objetos de estudo ideias para esses trabalhos iniciais.
    A maioria dos jogos utiliza um conjunto bem definido de regras: isso faz com que seja fácil gerar o espaço de busca e
    libera o pesquisador de muitas das ambiguidades e complexidades inerentes a problemas menos estruturados.
    As configurações do tabuleiro usadas nesses jogos são facilmente representáveis em um computador,
    dispensando o formalismo complexo necessário para capturar as sutilezas semânticas de domínios de problemas mais
    complexos. \cite[p.17]{luger}
\end{citacao}
\section{Relevância}
A inteligência artificial pode ser definida como sistemas ou máquinas que procuram imitar o raciocínio humano visando realizar
alguma atividade podendo aprimorar seu desempenho de forma interativa com base nas informações que coletam.

Com o constante aumento do seu uso na resolução de problemas dentro da sociedade, indo da construção de trajetos mais
eficientes para os serviços que proporcionam mapas até traçar um perfil de consumidor a partir da navegação de um usuário
dentro da internet, a inteligência artificial e seus algoritmos são grandes responsáveis em como o mundo funciona atualmente.

A implementação desses programas para a análise de dados tradicionais permite descrever problemas em que ações que já
aconteceram identificam novas oportunidades e implementam estratégias baseadas nos dados para chegar em um resultado
satisfatório.

Utilizando como referencial o jogo de xadrez para computador como exemplo,  é possível exemplificar os algoritmos que fazem
a base da inteligência artificial e mostrar como problemas podem ser representados de forma que uma máquina chegue a soluções
seguindo um raciocínio matemático comparável a logica humana.