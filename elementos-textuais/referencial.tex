\chapter{Referencial Teórico}
Nesta seção, serão apresentados os fundamentos teóricos que servirão de base para sustentar o estudo sobre a inteligência
artificial dentro da área de algoritmos e motores do jogo de xadrez online. Primeiramente, iremos compreender as teorias e
conceitos de base para esta pesquisa, que proveram as informações necessárias para a análise e comparação em questão.
Depois, falaremos sobre o estado da arte deste tema.

\section{Teoria e Conceitos de Base}
Esta parte do projeto conterá as informações teóricas necessárias para compreensão do tema e sua problematização,
assim veremos os conceitos de inteligência artificial, motor de xadrez, algoritmo de busca min-max,
algoritmo alpha-beta pruning e as funções de avaliação de otimização do tabuleiro por meio de algoritmos genéticos.

\subsection{Conceito de Inteligencia Artificial}
O conceito de inteligência artificial surgiu da ideia de reproduzir nas máquinas a capacidade humana de usar das informações
disponíveis para resolver problemas e tomar decisões com base na razão e lógica, o que resultou em dar aos computadores a
capacidade de automatizar processos ou pelo menos minimizar consideravelmente o envolvimento humano nós mesmos,
e com o avanço cada vez maior da velocidade de processamento das máquinas, elas alcançaram a capacidade de analisar dados
em taxas extremamente mais rápidas do que a humana.

Como dito por Hintze (2016),
\begin{citacao}
    Os tipos mais básicos de sistemas de IA são puramente reativos e não têm a capacidade de formar memórias nem de usar
    experiências passadas para informar as decisões atuais. Deep Blue, o supercomputador de xadrez da IBM, que derrotou o
    grande mestre internacional Garry Kasparov no final dos anos 1990, é o exemplo perfeito desse tipo de máquina.
    \cite[tradução nossa]{HINTZE}
\end{citacao}

É importante diferenciar o tipo mais básico de inteligência artificial dos mais complexos, pois o pensamento mais comum
quando falamos neste tema é a criação de máquinas semelhantes aos humanos que assim como nós possam pensar e agir por conta
própria, possuindo a capacidade de aprender e até mesmo possuir sentimentos e consciência, mas tais feitos só podem ser
alcançados utilizando-se de tecnologias de áreas como machine learning e redes neurais, que são ramos da inteligência
artificial.

\subsection{Conceito do motor de xadrez}

\subsection{Algoritmo de busca min-max}
O algoritmo minimax é aplicado em jogos adversariais e de soma zero,jogos esses que possuem 2 jogadores e cada um joga
por turnos, a vitória de um jogador implica na derrota do outro.

O minimax é um algoritmo de força bruta,isso significa que seu objetivo é enumerar todos os possíveis candidatos de uma
solução e verificar se cada um satisfaz o problema, o algoritmo divide as possibilidades de ações em uma árvore de jogadas
para conseguir a melhor jogada possível,essa árvore vai ser definida em etapas de minimização(min) e maximização(max),
sendo cada uma destas etapas representadas por uma jogada do adversário ou da máquina respectivamente,cada nó representa
uma configuração e cada aresta representa uma jogada que leva a uma determinada configuração.

O fim de uma aresta significa o fim de uma partida, nisso é aplicado uma avaliação para validar se aquele nó possui
um resultado positivo ou negativo para a máquina, o nó em questão recebe um valor com base no seu resultado.

Após o fim de todas as arestas de um nó subimos para o nó antecessor a essa aresta em questão, se o nó em questão for um nó
de min, ou seja,uma jogada do adversário, atribuímos ao nó o valor mínimo entre os valores de suas arestas, caso seja o max,
uma jogada da máquina, atribuímos o valor máximo entre suas arestas.

No fim de todas as arestas o algoritmo escolhe a aresta com o maior valor pois esta é a melhor alternativa para se seguir.

\subsection{Algoritmo alpha-beta pruning}

\subsection{Funções de avaliação de otimização do tabuleiro por meio de algoritmos genéticos}

\section{Estado da Arte}
